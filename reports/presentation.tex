\documentclass{beamer}


% Preamble for english writing
\usepackage[utf8]{inputenc}
\usepackage[T1]{fontenc}
\usepackage{amsmath}
\usepackage{amsfonts}
\usepackage{amssymb}
\usepackage{amsthm}
%\usepackage{algorithmic}
\usepackage{color}
\usepackage{enumerate}
\usepackage{stmaryrd}
\usepackage{hyperref}
\usepackage{graphicx}

%\usepackage{tkz-graph}

%\usepackage[nottoc, notlof, notlot]{tocbibind}

%%%%%%%%%%%%%%%%%%%%%%%%%%%%%%%%%%%%%%%%%%%%%%%%%%%%%%%%%%%%%%%%%%%
%Commandes ajoutées
%%%%%%%%%%%%%%%%%%%%%%%%%%%%%%%%%%%%%%%%%%%%%%%%%%%%%%%%%%%%%%%%%%%

%ensmembles
\newcommand{\N}{\mathbb{N}}
\newcommand{\Z}{\mathbb{Z}}
\newcommand{\Q}{\mathbb{Q}}
\newcommand{\R}{\mathbb{R}}


\newcommand{\intv}[2]{\llbracket #1, #2 \rrbracket} %intervalle d'entiers
\newcommand{\bracket}[1]{{\langle #1 \rangle}} %parenthèses angulaires
\newcommand{\gO}{\mathcal{O}} %grand O
\newenvironment{itemiz}{\renewcommand\labelitemi{$\blacktriangleright$}\begin{itemize}}{\end{itemize}}

\newcommand{\rg}{\text{rg}}
\newcommand{\bad}{\textbf{/!\backslash}}

\newcommand{\crk}{\text{cut-rank}}

\newcommand{\ie}{\emph{i.e.}\ }
\newcommand{\eg}{\emph{e.g.}\ }
\newcommand{\cf}{\emph{c.f.}\ }

% Theorems etc.
\newtheorem{Thm}{Theorem}[section]
\newtheorem{Cor}[Thm]{Corollary}
\newtheorem{Lem}[Thm]{Lemma}
\newtheorem{Pro}[Thm]{Proposition}

\theoremstyle{remark}
\newtheorem{Rem}[Thm]{Remark}
\newtheorem{Not}[Thm]{Notation}
\newtheorem{Exa}[Thm]{Example}

\theoremstyle{definition}
\newtheorem{Def}[Thm]{Definition}

\newcommand{\mytitle}[1]{
\begin{center}
\vspace{7cm}
\hrule
\vspace{0.5cm}
\huge{\textsc{#1}}
\vspace{0.5cm}
\hrule
\vspace{1cm}
\end{center}}



%\usecolortheme{beaver}
\usetheme{Berlin}

\title{\textbf{Evolutionary Computing \\on Turing Machines}}
\subtitle{Solving the Busy Beaver Problem}
\author{Jean-Florent Raymond \and Emilio Del Tessandoro}
\institute[Uppsala University]{Uppsala University}
\date{\today}



\begin{document}

\begin{frame}
\titlepage
%dont know if it's a good idea to add the university logo :
\vspace{-2cm}
\begin{flushright}
\includegraphics[scale=0.3]{figures/uu_logo.png}  
\end{flushright}

\end{frame}

\section{Introduction}

\begin{frame}{What Is a Turing Machine ?}


%they said to say briefly what it is, I think a picture is needed
\begin{figure}
\includegraphics[scale=0.5]{figures/turingMachine.png}
\end{figure}
\end{frame}

\begin{frame}{The Busy Beaver Problem}

\end{frame}

\begin{frame}{Why It Is So Hard}

%add another slide with the table, maybe two tables
%one fore the search space size, the other for the number of steps computed by the beavers
\begin{itemize}
\item The size is exponential in $n$, $m$
\item The halting problem, $S$ is not even computable
\item How the ``good'' machines are distributed is not known and not easy
\item $S(n,m)$ grows extremely fast
\end{itemize}
\end{frame}

\begin{frame}{Why It Is So Hard, Continued}


\only<1>{
\framesubtitle{Order of magnitude of the search space}

\begin{center}
\begin{tabular}[h!]{|c|c|c|c|c|}
  \hline
    & 2 symbols & 3 symbols  & 4 symbols\\
  \hline
  2 states&$10^4$ &$10^7$   &$10^{17}$\\
  \hline
  3 states&$10^7$ &$10^{12}$ &$10^{18}$\\
  \hline
  4 states&$10^8$ &$10^{17}$ &$10^{25}$\\
  \hline
\end{tabular}
\[
\left (2 \times M \times(N+1) \right ) ^{MN}
\]
\end{center}
}\only<2>{
\framesubtitle{Values and lower bounds for $S$}

\begin{center}
\begin{tabular}[h!]{|c|c|c|c|c|}
  \hline
          & 2 symbols & 3 symbols  & 4 symbols\\
  \hline
  2 states&6          &38          &$>10^6$\\
  \hline
  3 states&21         & $> 10^{17}$ &$>10^{13036}$\\
  \hline
  4 states&107        &$>10^{14072}$ &?\\
  \hline
\end{tabular}
\end{center}
}
\end{frame}

\section{Possible Solutions}

\begin{frame}{Other Approaches}

\begin{itemize}
\item Naïve method : run different machines and keep to one who halts after the more steps ; %improved if we detect loops
\item Detect patterns on the tape to predict future action.
\end{itemize}

\end{frame}

\begin{frame}{Our Approach}

We used a genetic algorithm.
\end{frame}

\begin{frame}{What We Did}

%C++ implementation, TM virtual machine and the EC tool
\begin{enumerate}
\item we did a C++ imlementation :
  \begin{itemize}
  \item program to handle and run TM ;
  \item genetic algorithm.
  \end{itemize}
\item we run tests ;
\item we get results.
\end{enumerate}
\end{frame}

\section{Results}

\begin{frame}{What We Found}
Busy beavers for the instances (2,2), (2,3), (3,2)
\end{frame}

\begin{frame}{Actual Results}
% we found all known busy beavers
\end{frame}

\section{Conclusions}

\begin{frame}{Conclusions}
We did a great work and then deserve the best grade, and maybe a Gödel prize.
\end{frame}

\end{document}