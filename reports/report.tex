\documentclass{report}


% Preamble for english writing
\usepackage[utf8]{inputenc}
\usepackage[T1]{fontenc}
\usepackage{amsmath}
\usepackage{amsfonts}
\usepackage{amssymb}
\usepackage{amsthm}
%\usepackage{algorithmic}
\usepackage{color}
\usepackage{enumerate}
\usepackage{stmaryrd}
\usepackage{hyperref}
\usepackage{graphicx}

%\usepackage{tkz-graph}

%\usepackage[nottoc, notlof, notlot]{tocbibind}

%%%%%%%%%%%%%%%%%%%%%%%%%%%%%%%%%%%%%%%%%%%%%%%%%%%%%%%%%%%%%%%%%%%
%Commandes ajoutées
%%%%%%%%%%%%%%%%%%%%%%%%%%%%%%%%%%%%%%%%%%%%%%%%%%%%%%%%%%%%%%%%%%%

%ensmembles
\newcommand{\N}{\mathbb{N}}
\newcommand{\Z}{\mathbb{Z}}
\newcommand{\Q}{\mathbb{Q}}
\newcommand{\R}{\mathbb{R}}


\newcommand{\intv}[2]{\llbracket #1, #2 \rrbracket} %intervalle d'entiers
\newcommand{\bracket}[1]{{\langle #1 \rangle}} %parenthèses angulaires
\newcommand{\gO}{\mathcal{O}} %grand O
\newenvironment{itemiz}{\renewcommand\labelitemi{$\blacktriangleright$}\begin{itemize}}{\end{itemize}}

\newcommand{\rg}{\text{rg}}
\newcommand{\bad}{\textbf{/!\backslash}}

\newcommand{\crk}{\text{cut-rank}}

\newcommand{\ie}{\emph{i.e.}\ }
\newcommand{\eg}{\emph{e.g.}\ }
\newcommand{\cf}{\emph{c.f.}\ }

% Theorems etc.
\newtheorem{Thm}{Theorem}[section]
\newtheorem{Cor}[Thm]{Corollary}
\newtheorem{Lem}[Thm]{Lemma}
\newtheorem{Pro}[Thm]{Proposition}

\theoremstyle{remark}
\newtheorem{Rem}[Thm]{Remark}
\newtheorem{Not}[Thm]{Notation}
\newtheorem{Exa}[Thm]{Example}

\theoremstyle{definition}
\newtheorem{Def}[Thm]{Definition}

\newcommand{\mytitle}[1]{
\begin{center}
\vspace{7cm}
\hrule
\vspace{0.5cm}
\huge{\textsc{#1}}
\vspace{0.5cm}
\hrule
\vspace{1cm}
\end{center}}




\title{Machine Learning Project Report\\\textbf{Evolutionnary Computing on\\Turing machines}}
\author{Jean-Florent Raymond\\\href{mailto:jean-florent.raymond.8795@student.uu.se}{\texttt{jean-florent.raymond.8795@student.uu.se}} \and Emilio Del Tessandoro\\
\href{mailto:emilio.del_tessandoro.4062@student.uu.se }{\texttt{emilio.del\_tessandoro.4062@student.uu.se }}}
\date{\today}

\begin{document}
\maketitle

\begin{abstract}
  Busy beavers are Turing machines that performs ``the most operations'' before halting among all Turing machines with the same alphabet and number of states. Finding busy beavers becomes quickly difficult when the alphabet size and the number of states grows because of the huge number of different machines and because of the halting problem, which make us unable to know if a given machine will halt or no. Currently, busy beavers are only known for small alphabets and number of states and have only be studied by detecting ``good'' patterns on the tape or by executing different machines until it halts (if so).
The goal of our project is to see if evolutionnary computing techiques are suitable for finding busy beaver. We deal with a population of Turing machines which can mutate an crossover. A big issue is the computation of the fitness function, since we don't know and cannot know when a machine will halt (if it halts).
\end{abstract}

\chapter{Introduction}
\label{chap:intro}

\section{Busy-beavers}
\label{chap:bb}
\paragraph{Busy Beaver and Max-step Functions}
Let $\sigma : \mathcal{T} \to \N \cup \{-\infty\}$ be the function which, given a Turing machine $M$, returns:
\begin{itemize}
\item either the number of non-blank symbols that $M$, started on a blank tape, left on the tape after execution if $M$ halts;
\item or minus infinity else.
\end{itemize}

Let the \emph{busy beaver function} $\Sigma$ be defined as follows:
\[
\Sigma : \left |
\begin{array}{l}
\N \times \N \to \N\\
(m,n) \mapsto \max_{M \in \mathcal{T}_{m,n}} \sigma(M)
\end{array}
\right .
\]

Similarly, let $s : \mathcal{T} \to \N \cup \{-\infty\}$ be the function which, given a Turing machine $M$, returns
\begin{itemize}
\item either the number of steps that $M$ performs when started on a blank tape, if $M$ halts ;
\item of infinity else.
\end{itemize}

And let \emph{max-step function} $S$ be defined by
\[
S : \left |
\begin{array}{l}
  \N \times \N \to \N\\
(m,n) \mapsto \max_{M \in \mathcal{T}_{m,n}} s(M)
\end{array}
\right .
\]

Note that these four function are not computable because otherwise it would means that we can decide the halting problem \cite{rado}. Moreover there exists some relations between $S$ and $\Sigma$; for example is immediate that $\forall m,n \in \N,\ S(m,n) \geq \Sigma(m,n)$, because writing $p$ non-blank symbols on the tape require at least the execution of $p$ steps. But there is no direct connection between $S$-busy beavers and $\Sigma$-busy beavers, \ie we cannot deduce one if we find the other one.

\section{Finding Machines That Reaches Bounds}

The goal of this project is to find, at least for small values of $m$ and $n$, Turing machines that reaches $\Sigma(m,n)$ or $S(m,n)$, \ie finding a machine $M \in \mathcal{T}_{m,n}$ such that $\sigma(M) = \Sigma(m,n)$ or $s(M) = S(m,n)$.

Such machines have already been found \cite{rado} but only for small values of $m$ and $n$ because $S$ and $\Sigma$ functions grows very fast \cite{rado}, thing that makes tests difficult.


\paragraph{How?} This will be done using evolutionary computing techniques on a population of Turing machines coming from the same class $\mathcal{T}_{m,n}$, for fixed $m$ and $n$. The operations of crossover and mutation are made from the transition tables (the alphabet and the number of states will remain unchanged).

After this, we also want to try an other approach using particle swarm optimization.

\paragraph{Fitness Function}
The main problem is to define a good fitness function. The value to maximize here is the number of steps before halting (for the max-step case), in order to reach the bound.

Suppose we have a machine $M$ which halts after $p$ steps and a machine $M'$ which continues to run after $p$ steps.
We cannot say that $M'$ is better than $M$ w.r.t.\ max-step, because we don't know if $M'$ will stop and we have no general way to know it (since the halting problem is undecidable \cite{turing}).
However, for fixed $p \in \N$, we are able to say if a given machine will stop in less than $p$ steps. To know it, we simply run the machine during $p$ steps and watch if it is on a halting state at this time.

The approach we want to try here is depends on what bound we want to reach:
\begin{itemize}
\item for small values of $n,m \in \N$, values of $\Sigma$ and $S$ are known \cite{rado} so a suitable fitness function is a function which, given a Turing machine $M$ that we already run on $p$ steps without leading it in the halting state, returns $- \infty$ if $p > S(m,n)$ and $\frac{1}{S(m,n) - p}$ else. With this function, Turing machines which overtake the bound are highly penalized (because we know they will never halt) and a machine which run longer (but less than $S(m,n)$) without halting than an other one get a better grade.

\item for values of $n, m \in \N$ for which $\Sigma$ and $S$ are not known, we can, for growing values of $p \in \N$, find machines which halts after $p$ steps with a fitness function of the same kind as before. Then the more $p$ will be high, the more we will approximate the bound if we still find machines for this $p$. Note that we will probalbly not able to use this approach for many different values of $m$ and $n$ because known lower bounds for $S$ and $\Sigma$ grows very fast, what make the computations long.
\end{itemize}

\section{Evolutionnary Computing}
\label{chap:ec}

\chapter{Strategy}
\label{chap:strategy}

\chapter{Work Organisation}
\label{chap:wo}

\chapter{Implementation}
\label{chap:impl}

\chapter{Results}
\label{chap:results}

\chapter{Outcome}
\label{chap:outcome}

\end{document}