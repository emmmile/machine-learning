\documentclass{report}


% Preamble for english writing
\usepackage[utf8]{inputenc}
\usepackage[T1]{fontenc}
\usepackage{amsmath}
\usepackage{amsfonts}
\usepackage{amssymb}
\usepackage{amsthm}
%\usepackage{algorithmic}
\usepackage{color}
\usepackage{enumerate}
\usepackage{stmaryrd}
\usepackage{hyperref}
\usepackage{graphicx}

%\usepackage{tkz-graph}

%\usepackage[nottoc, notlof, notlot]{tocbibind}

%%%%%%%%%%%%%%%%%%%%%%%%%%%%%%%%%%%%%%%%%%%%%%%%%%%%%%%%%%%%%%%%%%%
%Commandes ajoutées
%%%%%%%%%%%%%%%%%%%%%%%%%%%%%%%%%%%%%%%%%%%%%%%%%%%%%%%%%%%%%%%%%%%

%ensmembles
\newcommand{\N}{\mathbb{N}}
\newcommand{\Z}{\mathbb{Z}}
\newcommand{\Q}{\mathbb{Q}}
\newcommand{\R}{\mathbb{R}}


\newcommand{\intv}[2]{\llbracket #1, #2 \rrbracket} %intervalle d'entiers
\newcommand{\bracket}[1]{{\langle #1 \rangle}} %parenthèses angulaires
\newcommand{\gO}{\mathcal{O}} %grand O
\newenvironment{itemiz}{\renewcommand\labelitemi{$\blacktriangleright$}\begin{itemize}}{\end{itemize}}

\newcommand{\rg}{\text{rg}}
\newcommand{\bad}{\textbf{/!\backslash}}

\newcommand{\crk}{\text{cut-rank}}

\newcommand{\ie}{\emph{i.e.}\ }
\newcommand{\eg}{\emph{e.g.}\ }
\newcommand{\cf}{\emph{c.f.}\ }

% Theorems etc.
\newtheorem{Thm}{Theorem}[section]
\newtheorem{Cor}[Thm]{Corollary}
\newtheorem{Lem}[Thm]{Lemma}
\newtheorem{Pro}[Thm]{Proposition}

\theoremstyle{remark}
\newtheorem{Rem}[Thm]{Remark}
\newtheorem{Not}[Thm]{Notation}
\newtheorem{Exa}[Thm]{Example}

\theoremstyle{definition}
\newtheorem{Def}[Thm]{Definition}

\newcommand{\mytitle}[1]{
\begin{center}
\vspace{7cm}
\hrule
\vspace{0.5cm}
\huge{\textsc{#1}}
\vspace{0.5cm}
\hrule
\vspace{1cm}
\end{center}}




\title{Machine Learning Project Report\\\textbf{Evolutionnary Computing on\\Turing machines}}
\author{Jean-Florent Raymond\\\href{mailto:jean-florent.raymond.8795@student.uu.se}{\texttt{jean-florent.raymond.8795@student.uu.se}} \and Emilio Del Tessandoro\\
\href{mailto:emilio.del_tessandoro.4062@student.uu.se }{\texttt{emilio.del\_tessandoro.4062@student.uu.se }}}
\date{\today}

\begin{document}
\maketitle

\begin{abstract}
  Busy beavers are Turing machines that performs ``the most operations'' before halting among all Turing machines with the same alphabet and number of states. Finding busy beavers becomes quickly difficult when the alphabet size and the number of states grows because of the huge number of different machines and because of the halting problem, which make us unable to know if a given machine will halt or no. Currently, busy beavers are only known for small alphabets and number of states and have only be studied by detecting ``good'' patterns on the tape or by executing different machines until it halts (if so).
The goal of our project is to see if evolutionnary computing techiques are suitable for finding busy beaver. We deal with a population of Turing machines which can mutate an crossover. A big issue is the computation of the fitness function, since we don't know and cannot know when a machine will halt (if it halts).
\end{abstract}

\chapter{Introduction}
\label{chap:intro}

\section{Busy-beavers}
\label{chap:bb}

\section{Evolutionnary Computing}
\label{chap:ec}

\chapter{Strategy}
\label{chap:strategy}

\chapter{Implementation}
\label{chap:impl}

\chapter{Results}
\label{chap:results}

\chapter{Outcome}
\label{chap:outcome}

\end{document}