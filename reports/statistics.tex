\documentclass{report}


% Preamble for english writing
\usepackage[utf8]{inputenc}
\usepackage[T1]{fontenc}
\usepackage{amsmath}
\usepackage{amsfonts}
\usepackage{amssymb}
\usepackage{amsthm}
%\usepackage{algorithmic}
\usepackage{color}
\usepackage{enumerate}
\usepackage{stmaryrd}
\usepackage{hyperref}
\usepackage{graphicx}

%\usepackage{tkz-graph}

%\usepackage[nottoc, notlof, notlot]{tocbibind}

%%%%%%%%%%%%%%%%%%%%%%%%%%%%%%%%%%%%%%%%%%%%%%%%%%%%%%%%%%%%%%%%%%%
%Commandes ajoutées
%%%%%%%%%%%%%%%%%%%%%%%%%%%%%%%%%%%%%%%%%%%%%%%%%%%%%%%%%%%%%%%%%%%

%ensmembles
\newcommand{\N}{\mathbb{N}}
\newcommand{\Z}{\mathbb{Z}}
\newcommand{\Q}{\mathbb{Q}}
\newcommand{\R}{\mathbb{R}}


\newcommand{\intv}[2]{\llbracket #1, #2 \rrbracket} %intervalle d'entiers
\newcommand{\bracket}[1]{{\langle #1 \rangle}} %parenthèses angulaires
\newcommand{\gO}{\mathcal{O}} %grand O
\newenvironment{itemiz}{\renewcommand\labelitemi{$\blacktriangleright$}\begin{itemize}}{\end{itemize}}

\newcommand{\rg}{\text{rg}}
\newcommand{\bad}{\textbf{/!\backslash}}

\newcommand{\crk}{\text{cut-rank}}

\newcommand{\ie}{\emph{i.e.}\ }
\newcommand{\eg}{\emph{e.g.}\ }
\newcommand{\cf}{\emph{c.f.}\ }

% Theorems etc.
\newtheorem{Thm}{Theorem}[section]
\newtheorem{Cor}[Thm]{Corollary}
\newtheorem{Lem}[Thm]{Lemma}
\newtheorem{Pro}[Thm]{Proposition}

\theoremstyle{remark}
\newtheorem{Rem}[Thm]{Remark}
\newtheorem{Not}[Thm]{Notation}
\newtheorem{Exa}[Thm]{Example}

\theoremstyle{definition}
\newtheorem{Def}[Thm]{Definition}

\newcommand{\mytitle}[1]{
\begin{center}
\vspace{7cm}
\hrule
\vspace{0.5cm}
\huge{\textsc{#1}}
\vspace{0.5cm}
\hrule
\vspace{1cm}
\end{center}}



\usepackage{pgf}
\usepackage{pgfplots}

\pgfplotsset{
	legend style={rounded corners=2pt},
}

\title{Machine Learning Project Questions\\\textbf{Evolutionary Computing\\on Turing machines}\\{\Large Solving the Busy Beaver Problem}}
\author{Jean-Florent Raymond\\\href{mailto:jean-florent.raymond.8795@student.uu.se}{\texttt{jean-florent.raymond.8795@student.uu.se}} \and Emilio Del Tessandoro\\
  \href{mailto:emilio.del_tessandoro.4062@student.uu.se }{\texttt{emilio.del\_tessandoro.4062@student.uu.se }}}
\date{\today}

\begin{document}
\maketitle

\section*{Question 1}

\emph{``The results are somewhat brief; it's good that you were able to find
    the known busy beavers, but you don't give an idea how difficult it
    is to locate them. How does the performance of your algorithm
    compare with a simple random search? You could emulate random
    searching by turning off crossover and using mutation only to
    explore the space, and then give a comparison as to how much more
    quickly your algorithm converged to the best solution.''}

\section*{Answer to Question 1 : Random Search \emph{vs} Genetic Algorithm}

By random search, we mean randomly drawing a new Turing machine until we find a busy beaver.
We present here the measures of the time taken by finding a busy beaver with random search versus finding a busy beaver with the genetic algorithm we wrote for the project. Sadly, these measures shows that random search is faster. Doing these benchmarks before writing the report would have change the conclusion: GA seems not suitable for efficiently finding busy beavers. We found some, probably because the search space is small enough to make busy beavers appears in a population in a reasonable time. Results for the case $n = 3$ and $m = 2$ are on page \pageref{plot:rnd}, figure \ref{plot:rnd}.

\begin{figure}[t]
\centering
\begin{tikzpicture}
    \begin{semilogyaxis}[
        xlabel=executions,
        ylabel=time (s),
	xmin=0,
	xmax=3,
	legend pos=south east]
        \addplot[only marks,color=red] table[x=exp,y=t] {../results/times};
	\addlegendentry{normal}
	\addplot[only marks, color=blue]   table[x=rnd,y=time] {../results/times};
	\addlegendentry{random search}
    \end{semilogyaxis}
\end{tikzpicture}
\caption[]{Statistics for the case $N=3$, $M=2$.}
\label{plot:rnd}
\end{figure}




\section*{Question 2}
\emph{``The best individuals seem to be highly separated, and even small
    changes can make a strong contender into a weak one. That fact
    doesn't make PSO seem like a good choice of meta-heuristic (the
    swarm would likely have trouble 'knowing' when it had found a
    promising area); at the same time, it makes designing a good
    crossover operator challenging. Evolutionary programming might be a
    better way to go here, especially since it doesn't use crossover
    (just mutation) -- it would have been interesting to see a
    discussion of the possible merits of EP as applied to this problem.''}

\section*{Answer to Question 2}

In this problem, we hoped that the application of genetic operators on individuals will lead to better individuals, and at the end, to a solution (a busy beaver).
We used two different genetic operators:
\begin{enumerate}
\item the mutation. This is the simplest genetic operator and there are no different ways to do it\footnote{We could ``play'' with different random distributions but it's not assured that selecting a small numerical mutation we end up in a machine with a behaviour similar to the parent (and vice versa, if the mutation is numerically big). The same argument can be applied when speaking about mutating multiple points in the genome.}, we just randomly changed symbols in the transition table. We know that it may lead to a better solution since it randomly changes the genome, so it may mutate it to the genome of a better individual (w.r.t\ the fitness function).
\item the crossover. This is a genetic operator inspired by nature but we don't know:
  \begin{itemize}
  \item if it is relevant in this problem (and why);
  \item what kind of crossover to use (\texttt{ONE\_POINT}, \texttt{TWO\_POINT}, \texttt{UNIFORM}, or other types).
  \end{itemize}
\end{enumerate}

\paragraph{Defining Crossover}
The most difficult problem, after deciding a way to compute the fitness, is here to choose a good crossover operator. We have no way to know how we should mix the transition tables to obtain a good machine: crossover may be meaningful when the two parents execution follows close steps but in the other (and much more probable) cases, it is more like a weird mutation than a way to take the best from the parents.
We used crossover because it is always used in genetic algorithms and it is closer to what nature does. In fact, since it seems meaningless, it is probably better to remove crossover. 
Without this operator, this way of doing (genetic algorithm) becomes closer to evolutionary programming.

\paragraph{Genome Structure}
The genome of our individuals does not follows the linear representation usually used in genetic algorithms, it is the transition table directly coming from the definition of a Turing machine. In this way it is closer to evolutionary programming than to common genetic algorithms.

\paragraph{Selection}
Evolutionary programming normally uses a $(\mu + \mu)$ tournament selection \cite{introtoec}, that is deterministic, but often is used in a stochastic variation as follows. From a population of $\mu$ individuals generate by mutations a offspring of other $\mu$ new individuals. From the whole population of $2\mu$ individuals select the fittest $\mu$ using a rank-based tournament procedure: every individual is compared (in terms of fitness value) with other $q$ randomly selected individuals. The individuals who ``win'' the most after these tournaments are the selected ones. 

So this is an elitism strategy (the best individuals for sure will have a high score after the tournaments) and it is actually quite similar to what we implemented (elitism on a certain number of individuals and stochastic selection on the rest). At the end this shouldn't change at all the general trend of the experiments we have done in the previous section.

\paragraph{Mutations}
Evolutionary programming often uses a variable mutation strategy: the closer the individual is to the solution, the less important is the mutation. Since we did not run tests with evolutionary programming, it is hard to say if modulating mutations is helpful in the problem we are interested in. Nevertheless, we can say two things:
%I don't agree too much on this point. I'm quite sure that small mutations doesn't imply close fitness values (it is the point they asked us to clear): Joe said "and even small changes can make a strong contender into a weak one"..
\begin{enumerate}
\item using small mutations when the individual is close to the optimum is good because at least we are sure that in the general case, a strong mutation will not improve the fitness;
\item using strong mutations when the individual is far away to the optimum can be seen as deleting the individual and randomly drawing a new one; in this way it operates an elitist selection which is good.
\end{enumerate}


\section*{An Accidental EP Implementation}

When trying to run random search versus GA, we first set the crossover probability to 0. Later, we thougth it was not pure random search, but something close to evolutionnary programming. We present here the results we found.

The only cases we tested are:
\begin{itemize}
\item $N = 3$, $M = 2$ (plots \ref{plot32} on page \pageref{plot32} and \ref{plot32good} on page \pageref{plot32good}),
\item $N = 4$, $M = 2$ (plot \ref{plot42} on page \pageref{plot42}),
\item $N = 2$, $M = 3$ (plots \ref{plot23} on page \pageref{plot23}),
\end{itemize}

that are the most difficult ones where the upper bound of the $S$ function is known. The cases where the upper bound is not known are too difficult to solve and hasn't been investigated.


\subsection*{With Non-Stationary Fitness Function}
Plot \ref{plot32} show a simple comparison of the two approaches. %In this case the best machine(s) computes 21 steps and the search space size is 16\,777\,216.
Fixed the generations number, the EP obviously explore less compared to the normal evolutionary approach (because no crossovers are performed), so the normal approach should be always more effective. In fact the EP produces on average one third of the machines produced by the plain evolutionary approach (using the same number of generations and a crossover probability of 0.9).
But also considering this fact, the EP is instead more effective.
This happens because it's true that crossovers introduce more ``good'' machines, but is also true that, since the population size is limited, these machines are selected in preference of other ones (machines that instead are kept in the EP).

A more fair comparison is shown in plot \ref{plot32good} (on page \pageref{plot32good}). This time the number of steps of the best machine found (average) is plotted against the number of explored machines (average)\footnote{This is not exact. The number of explored machines is strictly less than the shown numbers because we can generate more times the same machine (we didn't add a check for this fact).}. In this case is evident that EP is always better.

Notice that EP is better not only in terms of the ratio ``how good is the found machine'' versus ``number of explored machines'', but also in time complexity. This because producing a new machine via crossover can take time proportional to the transition table size (\ie $O(MN)$), while a mutation is always $O(1)$ in our implementation.

\begin{figure}[t]
\centering
\begin{tikzpicture}
    \begin{semilogxaxis}[
        xlabel=generations,
        ylabel=best found (average),
	xmin=10,
	xmax=1000,
	legend pos=south east]
        
	\addplot[smooth,color=red]   table[x=generations,y=averageSteps] {../results/statistics-non-stationary/3.2.statistics};
	\addlegendentry{normal}
	\addplot[smooth,color=blue] table[x=generations,y=averageSteps] {../results/statistics-non-stationary/3.2.statistics.random};
	\addlegendentry{only mutations}
    \end{semilogxaxis}
\end{tikzpicture}
\caption[]{Statistics for the case $N=3$, $M=2$. Best machine found (an average over 200 program executions) versus the generations number. This comparison it's not fair since the normal algorithm produces a lot more machines than the (only) random version. But, also considering this fact, the plot shows that the EP is more effective.}
\label{plot32}
\end{figure}


\begin{figure}[b]
\centering
\begin{tikzpicture}
    \begin{semilogxaxis}[
        xlabel=explored machines,
        ylabel=best found (average),
	%width=\textwidth,
	xmin=3000,
	xmax=600000,
	legend pos=south east]
		
	\addplot[smooth,color=red]   table[x=averageExplored,y=averageSteps] {../results/statistics-non-stationary/3.2.statistics};
	\addlegendentry{normal}
	\addplot[smooth,color=blue] table[x=averageExplored,y=averageSteps] {../results/statistics-non-stationary/3.2.statistics.random};
	\addlegendentry{only mutations}
    \end{semilogxaxis}
\end{tikzpicture}
\caption[]{Statistics for the case $N=3$, $M=2$. Best machine found (an average over 200 program executions) versus the number of explored machines (average). This is a more fair comparison w.r.t the previous plot.}
\label{plot32good}
\end{figure}




\begin{figure}[h]
\centering
\begin{tikzpicture}
    \begin{semilogxaxis}[
        xlabel=explored machines,
        ylabel=best found (average),
	%width=\textwidth,
	xmin=3000,
	xmax=2000000,
	legend pos=south east]
		
	\addplot[smooth,color=red]   table[x=averageExplored,y=averageSteps] {../results/statistics-non-stationary/4.2.statistics};
	\addlegendentry{normal}
	\addplot[smooth,color=blue] table[x=averageExplored,y=averageSteps] {../results/statistics-non-stationary/4.2.statistics.random};
	\addlegendentry{only mutations}
    \end{semilogxaxis}
\end{tikzpicture}
\caption[]{Statistics for the case $N=4$, $M=2$.}
\label{plot42}
\end{figure}


\begin{figure}[h]
\centering
\begin{tikzpicture}
    \begin{semilogxaxis}[
        xlabel=explored machines,
        ylabel=best found (average),
	%width=\textwidth,
	xmin=3000,
	xmax=2000000,
	legend pos=south east]
		
	\addplot[smooth,color=red]   table[x=averageExplored,y=averageSteps] {../results/statistics-non-stationary/2.3.statistics};
	\addlegendentry{normal}
	\addplot[smooth,color=blue] table[x=averageExplored,y=averageSteps] {../results/statistics-non-stationary/2.3.statistics.random};
	\addlegendentry{only mutations}
    \end{semilogxaxis}
\end{tikzpicture}
\caption[]{Statistics for the case $N=2$, $M=3$. Exploring $10^6$ machines correspond to use about 3000 generations in the EP: with these settings an optimal machine is found 80\% of the times.}
\label{plot23}
\end{figure}






\subsection*{With Stationary Fitness Function}
The tests made until now use a non stationary fitness function that, in the first generations, compute only a certain number of steps on the machine, increasing this number of steps in the following generations.

The situation becomes slightly better using a stationary fitness function, approach that is possible since we know the value of the $S$ function for these cases (figures \ref{plot32stationary} on page \pageref{plot32stationary}, \ref{plot42stationary} on page \pageref{plot42stationary}, \ref{plot23stationary} on page \pageref{plot23stationary}). The efficiency of the algorithm, in terms of how many generations are needed for finding a machine that compute $n$ steps, is higher in both cases. Also the gap between the two approaches becomes smaller.

An interesting situation is shown in figure \ref{plot23stationary} (on page \pageref{plot23stationary}) where, for some settings, the evolutionary approach is better than the EP.

\begin{figure}[h]
\centering
\begin{tikzpicture}
    \begin{semilogxaxis}[
        xlabel=explored machines,
        ylabel=best found (average),
	%width=\textwidth,
	xmin=3000,
	xmax=600000,
	legend pos=south east]
		
	\addplot[smooth,color=red]   table[x=averageExplored,y=averageSteps] {../results/statistics-stationary/3.2.statistics};
	\addlegendentry{normal}
	\addplot[smooth,color=blue] table[x=averageExplored,y=averageSteps] {../results/statistics-stationary/3.2.statistics.random};
	\addlegendentry{only mutations}
    \end{semilogxaxis}
\end{tikzpicture}
\caption[]{Statistics for the case $N=3$, $M=2$, stationary fitness function.}
\label{plot32stationary}
\end{figure}


\begin{figure}[h]
\centering
\begin{tikzpicture}
    \begin{semilogxaxis}[
        xlabel=explored machines,
        ylabel=best found (average),
	%width=\textwidth,
	xmin=3000,
	xmax=1000000,
	legend pos=south east]
		
	\addplot[smooth,color=red]   table[x=averageExplored,y=averageSteps] {../results/statistics-stationary/4.2.statistics};
	\addlegendentry{normal}
	\addplot[smooth,color=blue] table[x=averageExplored,y=averageSteps] {../results/statistics-stationary/4.2.statistics.random};
	\addlegendentry{only mutations}
    \end{semilogxaxis}
\end{tikzpicture}
\caption[]{Statistics for the case $N=4$, $M=2$, stationary fitness function. The idea is the same as figure \ref{plot32stationary} except that here the average is made over 500 program executions.}
\label{plot42stationary}
\end{figure}


\begin{figure}[h]
\centering
\begin{tikzpicture}
    \begin{semilogxaxis}[
        xlabel=explored machines,
        ylabel=best found (average),
	%width=\textwidth,
	xmin=3000,
	xmax=3000000,
	legend pos=south east]
		
	\addplot[smooth,color=red]   table[x=averageExplored,y=averageSteps] {../results/statistics-stationary/2.3.statistics};
	\addlegendentry{normal}
	\addplot[smooth,color=blue] table[x=averageExplored,y=averageSteps] {../results/statistics-stationary/2.3.statistics.random};
	\addlegendentry{only mutations}
    \end{semilogxaxis}
\end{tikzpicture}
\caption[]{Statistics for the case $N=2$, $M=3$, stationary fitness function. Again what is shown is an average over 500 program executions. This is the only case where the normal approach shows some advantages in respect to the pure EP; but increasing more the number of generations EP becomes better again.}
\label{plot23stationary}
\end{figure}



To conclude, evolutionary programming seems a better choice for solving this problem: it avoids the difficult definition of crossover and the mutation modulation may be helpful.

\bibliographystyle{plain}
\bibliography{statistics}

\end{document}