\documentclass{article}


% Preamble for english writing
\usepackage[utf8]{inputenc}
\usepackage[T1]{fontenc}
\usepackage{amsmath}
\usepackage{amsfonts}
\usepackage{amssymb}
\usepackage{amsthm}
%\usepackage{algorithmic}
\usepackage{color}
\usepackage{enumerate}
\usepackage{stmaryrd}
\usepackage{hyperref}
\usepackage{graphicx}

%\usepackage{tkz-graph}

%\usepackage[nottoc, notlof, notlot]{tocbibind}

%%%%%%%%%%%%%%%%%%%%%%%%%%%%%%%%%%%%%%%%%%%%%%%%%%%%%%%%%%%%%%%%%%%
%Commandes ajoutées
%%%%%%%%%%%%%%%%%%%%%%%%%%%%%%%%%%%%%%%%%%%%%%%%%%%%%%%%%%%%%%%%%%%

%ensmembles
\newcommand{\N}{\mathbb{N}}
\newcommand{\Z}{\mathbb{Z}}
\newcommand{\Q}{\mathbb{Q}}
\newcommand{\R}{\mathbb{R}}


\newcommand{\intv}[2]{\llbracket #1, #2 \rrbracket} %intervalle d'entiers
\newcommand{\bracket}[1]{{\langle #1 \rangle}} %parenthèses angulaires
\newcommand{\gO}{\mathcal{O}} %grand O
\newenvironment{itemiz}{\renewcommand\labelitemi{$\blacktriangleright$}\begin{itemize}}{\end{itemize}}

\newcommand{\rg}{\text{rg}}
\newcommand{\bad}{\textbf{/!\backslash}}

\newcommand{\crk}{\text{cut-rank}}

\newcommand{\ie}{\emph{i.e.}\ }
\newcommand{\eg}{\emph{e.g.}\ }
\newcommand{\cf}{\emph{c.f.}\ }

% Theorems etc.
\newtheorem{Thm}{Theorem}[section]
\newtheorem{Cor}[Thm]{Corollary}
\newtheorem{Lem}[Thm]{Lemma}
\newtheorem{Pro}[Thm]{Proposition}

\theoremstyle{remark}
\newtheorem{Rem}[Thm]{Remark}
\newtheorem{Not}[Thm]{Notation}
\newtheorem{Exa}[Thm]{Example}

\theoremstyle{definition}
\newtheorem{Def}[Thm]{Definition}

\newcommand{\mytitle}[1]{
\begin{center}
\vspace{7cm}
\hrule
\vspace{0.5cm}
\huge{\textsc{#1}}
\vspace{0.5cm}
\hrule
\vspace{1cm}
\end{center}}



\usepackage{pgfgantt} % for gantt diagram
\usepackage{rotating} % for rotating gantt diagram

\title{Machine Learning Project Proposal\\\textbf{Evolutionnary Computing on\\Turing machines}}
\author{Jean-Florent Raymond\\\href{mailto:jean-florent.raymond.8795@student.uu.se}{\texttt{jean-florent.raymond.8795@student.uu.se}} \and Emilio Del Tessandoro\\
\href{mailto:emilio.del_tessandoro.4062@student.uu.se }{\texttt{emilio.del\_tessandoro.4062@student.uu.se }}}
\date{\today}

\begin{document}
\maketitle

\section{The Busy Beaver}

\paragraph{Introduction} Given an number of states and a number of symbols, a \emph{busy beaver} is a deterministic Turing machine that, when started on a blank tape, performs ``the most operations'' before halting among all Turing machines with the same numbers of states and symbols. Performing ``the most operations'' can be definded in many ways, for instance it can mean:
\begin{itemize}
\item doing the most steps;
\item stopping with the most non-blank symbols written on the tape.
\end{itemize}

All the Turing machines we will consider here are deterministic. Let $\mathcal{T}_{m,n}$ be the set of deterministic Turing machines with an alphabet of $m$ symbols which uses $n + 1$ states, one of whom is the halt state, reached at most one time (if the machine halts). Let  $\mathcal{T} = \cup_{n,m \in \N} \mathcal{T}_{n,m}$.


\paragraph{Busy Beaver and Max-step Functions}
Let $\sigma : \mathcal{T} \to \N \cup \{-\infty\}$ be the function which, given a Turing machine $M$, returns:
\begin{itemize}
\item either the number of non-blank symbols that $M$, started on a blank tape, left on the tape after execution if $M$ halts;
\item or minus infinity else.
\end{itemize}

Let the \emph{busy beaver function} $\Sigma$ be defined as follows:
\[
\Sigma : \left |
\begin{array}{l}
\N \times \N \to \N\\
(m,n) \mapsto \max_{M \in \mathcal{T}_{m,n}} \sigma(M)
\end{array}
\right .
\]

Similarly, let $s : \mathcal{T} \to \N \cup \{-\infty\}$ be the function which, given a Turing machine $M$, returns
\begin{itemize}
\item either the number of steps that $M$ performs when started on a blank tape, if $M$ halts ;
\item of infinity else.
\end{itemize}

And let \emph{max-step function} $S$ be defined by
\[
S : \left |
\begin{array}{l}
  \N \times \N \to \N\\
(m,n) \mapsto \max_{M \in \mathcal{T}_{m,n}} s(M)
\end{array}
\right .
\]

Note that these four function are not computable because otherwise it would means that we can decide the halting problem \cite{rado}. Moreover there exists some relations between $S$ and $\Sigma$; for example is immediate that $\forall m,n \in \N,\ S(m,n) \geq \Sigma(m,n)$, because writing $p$ non-blank symbols on the tape require at least the execution of $p$ steps. But there is no direct connection between $S$-busy beavers and $\Sigma$-busy beavers, \ie we cannot deduce one if we find the other one.

\section{Finding Machines That Reaches Bounds}

The goal of this project is to find, at least for small values of $m$ and $n$, Turing machines that reaches $\Sigma(m,n)$ or $S(m,n)$, \ie finding a machine $M \in \mathcal{T}_{m,n}$ such that $\sigma(M) = \Sigma(m,n)$ or $s(M) = S(m,n)$.

Such machines have already been found \cite{rado} but only for small values of $m$ and $n$ because $S$ and $\Sigma$ functions grows very fast \cite{rado}, thing that makes tests difficult.


\paragraph{How?} This will be done using evolutionary computing techniques on a population of Turing machines coming from the same class $\mathcal{T}_{m,n}$, for fixed $m$ and $n$. The operations of crossover and mutation are made from the transition tables (the alphabet and the number of states will remain unchanged).

After this, we also want to try an other approach using particle swarm optimization.

\paragraph{Fitness Function}
The main problem is to define a good fitness function. The value to maximize here is the number of steps before halting (for the max-step case), in order to reach the bound.

Suppose we have a machine $M$ which halts after $p$ steps and a machine $M'$ which continues to run after $p$ steps.
We cannot say that $M'$ is better than $M$ w.r.t.\ max-step, because we don't know if $M'$ will stop and we have no general way to know it (since the halting problem is undecidable \cite{turing}).
However, for fixed $p \in \N$, we are able to say if a given machine will stop in less than $p$ steps. To know it, we simply run the machine during $p$ steps and watch if it is on a halting state at this time.

The approach we want to try here is depends on what bound we want to reach:
\begin{itemize}
\item for small values of $n,m \in \N$, values of $\Sigma$ and $S$ are known \cite{rado} so a suitable fitness function is a function which, given a Turing machine $M$ that we already run on $p$ steps without leading it in the halting state, returns $- \infty$ if $p > S(m,n)$ and $\frac{1}{S(m,n) - p}$ else. With this function, Turing machines which overtake the bound are highly penalized (because we know they will never halt) and a machine which run longer (but less than $S(m,n)$) without halting than an other one get a better grade.

\item for values of $n, m \in \N$ for which $\Sigma$ and $S$ are not known, we can, for growing values of $p \in \N$, find machines which halts after $p$ steps with a fitness function of the same kind as before. Then the more $p$ will be high, the more we will approximate the bound if we still find machines for this $p$. Note that we will probalbly not able to use this approach for many different values of $m$ and $n$ because known lower bounds for $S$ and $\Sigma$ grows very fast, what make the computations long.
\end{itemize}

\section{Work  Packages}

The project will be split in several work packages:
\begin{enumerate}
\item Virtual Machine 
  \begin{itemize}
  \item defining a convenient way to represent Turing machines;
  \item writing a basic virtual machine to run Turing machines;
  \item adding the ability to load and save machine states in a file and to log activity;
  \item adding abilities of step counting and of running until the machine halts (if it halts).
  \end{itemize}
\item Evolutionary Computing Tool
  \begin{itemize}
  \item defining operations on Turing machines (crossover, mutations);
  \item defining a fitness function;
  \item writing a program which generates a random initial population and which follows it until it leads to something interesting.
  \end{itemize}
\item Particle Swarm Optimization Tool
  \begin{itemize}
  \item discuss feasibility of such approach, eventually defining the search space and differences with the normal PSO approach;
  \item implementation related to the Turing machines work previously realized. 
  \end{itemize}
\item Testing
  \begin{itemize}
  \item testing all the parts of the program we write the ensure that we will not do hours of useless computation during the experiments;
  \item launching the experiments.
  \end{itemize}
\item Analyse
  \begin{itemize}
  \item analysing the results: Do we found what we were looking for? Do we discover other things?
  \item analysing the approach: Was evolutionary computing a good idea for this problem, and why?
  \end{itemize}
\item Writing documentation.
\end{enumerate}

%I agree on everything. I would try also a PSO approach if we can manage to work with the discrete search space (the binary PSO is an example).

The schedule for this project can be found on figure \ref{fig:sc}, page \pageref{fig:sc}.

\begin{sidewaysfigure}[p]
  \centering
  
  \begin{tikzpicture}[scale = 0.8]
    \begin{ganttchart}[vgrid]{37}
      \gantttitle{April}{8}\gantttitle{May}{29} \\
      \gantttitlelist{17,...,21}{7}\gantttitlelist{22}{2}\\
      \gantttitlelist[title label font={\tiny}]{23,...,30}{1}
      \gantttitlelist[title label font={\tiny}]{1,...,29}{1}\\

      \ganttgroup{Virtual Machine}{1}{19} \\
      \ganttbar{Defining the encoding}{1}{4} \\
      \ganttbar{Basic VM}{5}{13} \\
      \ganttbar{Adding functionalities}{14}{19} \\

      \ganttgroup{Evolutionary Computing Tool}{1}{19} \\
      \ganttbar{Defining the operations, etc.}{1}{3} \\
      \ganttbar{Implementing the tool}{4}{19} \\
      
      \ganttgroup{PSO Tool}{8}{19} \\
      \ganttbar{Discuss and define such approach}{8}{12} \\
      \ganttbar{Implementing the tool}{13}{19} \\

      \ganttgroup{Testing}{1}{37} \\
      \ganttbar{Testing our code}{1}{37} \\
      \ganttbar{Launching the experiments}{20}{37} \\

      \ganttgroup{Writing Documentation}{1}{37}

    \end{ganttchart}  
  \end{tikzpicture}

  \caption{Schedule}
  \label{fig:sc}
\end{sidewaysfigure}




\section{Related Work}
The work of Shen Lin and Tibor Rado on busy bever was based on reducing the space search with normalization techniques and then on examining patterns on the tape \cite{rado}.

\bibliographystyle{acm}
\bibliography{proposal_turing}

\end{document}